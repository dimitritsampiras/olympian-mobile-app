\documentclass[12pt, titlepage]{article}

\usepackage{fullpage}
\usepackage[round]{natbib}
\usepackage{multirow}
\usepackage{booktabs}
\usepackage{tabularx}
\usepackage{graphicx}
\usepackage{float}
\usepackage{hyperref}
\hypersetup{
	colorlinks,
	citecolor=blue,
	filecolor=black,
	linkcolor=red,
	urlcolor=blue
}

\input{../../Comments}

\newcommand{\progname}{Software Eng 4G06} % PUT YOUR PROGRAM NAME HERE
\newcommand{\authname}{Team 2, Parnas' Pals
	\\ William Lee
	\\ Jared Bentvelsen
	\\ Bassel Rezkalla
	\\ Yuvraj Randhawa
	\\ Dimitri Tsampiras
	\\ Matthew McCracken} % AUTHOR NAMES                  

\usepackage{hyperref}
\hypersetup{colorlinks=true, linkcolor=blue, citecolor=blue, filecolor=blue,
	urlcolor=blue, unicode=false}
\urlstyle{same}



\newcounter{acnum}
\newcommand{\actheacnum}{AC\theacnum}
\newcommand{\acref}[1]{AC\ref{#1}}

\newcounter{ucnum}
\newcommand{\uctheucnum}{UC\theucnum}
\newcommand{\uref}[1]{UC\ref{#1}}

\newcounter{mnum}
\newcommand{\mthemnum}{M\themnum}
\newcommand{\mref}[1]{M\ref{#1}}

\begin{document}
	
	\title{System Design for \progname{}} 
	\author{\authname}
	\date{\today}
	
	\maketitle
	
	\pagenumbering{roman}
	
	\section{Revision History}
	
	\begin{tabularx}{\textwidth}{p{3cm}p{2cm}X}
		\toprule {\bf Date} & {\bf Version} & {\bf Notes}\\
		\midrule
		Date 1 & 1.0 & Notes\\
		Date 2 & 1.1 & Notes\\
		\bottomrule
	\end{tabularx}
	
	\newpage
	
	\section{Reference Material}
	
	This section records information for easy reference.
	
	\subsection{Abbreviations and Acronyms}
	
	\renewcommand{\arraystretch}{1.2}
	\begin{tabular}{l l} 
		\toprule		
		\textbf{symbol} & \textbf{description}\\
		\midrule 
		\progname & Explanation of program name\\
		\wss{...} & \wss{...}\\
		\bottomrule
	\end{tabular}\\
	
	\newpage
	
	\tableofcontents
	
	\newpage
	
	\listoftables
	
	\listoffigures
	
	\newpage
	
	\pagenumbering{arabic}
	
	\section{Introduction}
	
	\wss{Include references to your other documentation}
	
	\section{Purpose}
	
	\wss{Purpose of your design documentation}
	
	\wss{Point to your other design documents}
	
	\section{Scope}
	
	\wss{Include a figure that show the System Context (showing the boundary between
		your system and the environment around it.)}
	
	\section{Project Overview}
	
	\subsection{Normal Behaviour}

	\subsection{Undesired Event Handling}

	\subsection{Component Diagram}
	
	\subsection{Connection Between Requirements and Design} \label{SecConnection}
	
	\begin{tabular}{|c|c|}
		\hline
		Requirement & Design \\
		\hline
		The product shall appear minimal and straightforward & The application will contain an interface that only presents information that is necessary as to prevent cluttering and reduce minimality \\
		\hline
		The product shall use fonts of readable size to the target user group & Fonts of size 10-12 will be used across the application \\
		\hline
		The product shall be able to be used by untrained fitness enthusiasts and amateurs alike, who receive no training before using it & UI will be simplistic and consistent throughout, reducing the learning curve for users \\
		\hline
		The product shall be usable by users with hearing loss or partial blindness & The application will rely on both audio and visual cues to indicate correct inputs or application events \\
		\hline
		The application must inform users when maintenance is taking place and must warn them at least 1 day in advance & The application will utilize pop up screens to display important notifications to the user \\
		\hline
		The applicaiton will allow users to report offensive content and remove it from their feed & There will be a button on each post that gives user the option to report offensive content \\
		\hline

	\end{tabular}
	
	\section{System Variables}
	
	\wss{Include this section for Mechatronics projects}
	
	\subsection{Monitored Variables}
	
	\subsection{Controlled Variables}
	
	\subsection{Constants Variables}
	
	\section{User Interfaces}
	
	\wss{Design of user interface for software and hardware.  Attach an appendix if
		needed. Drawings, Sketches, Figma}
	
	\section{Design of Hardware}
	
	\wss{Most relevant for mechatronics projects}
	\wss{Show what will be acquired}
	\wss{Show what will be built, with detail on fabrication and materials}
	\wss{Include appendices as appropriate, possibly with sketches, drawings, CAD, etc}
	
	\section{Design of Electrical Components}
	
	\wss{Most relevant for mechatronics projects}
	\wss{Show what will be acquired}
	\wss{Show what will be built, with detail on fabrication and materials}
	\wss{Include appendices as appropriate, possibly with sketches, drawings,
		circuit diagrams, etc}
	
	\section{Design of Communication Protocols}
	
	\wss{If appropriate}
	
	\section{Timeline}
	
	\wss{Schedule of tasks and who is responsible}
	
	% \bibliographystyle {plainnat}
	% \bibliography{../../../refs/References}
	
	\newpage{}
	
	\appendix
	
	\section{Interface}
	
	\wss{Include additional information related to the appearance of, and
		interaction with, the user interface}
	
	\section{Mechanical Hardware}
	
	\section{Electrical Components}
	
	\section{Communication Protocols}
	
	\section{Reflection}
	
	The information in this section will be used to evaluate the team members on the
	graduate attribute of Problem Analysis and Design.  Please answer the following questions:
	
	\begin{enumerate}
		\item What are the limitations of your solution?  Put another way, given
		unlimited resources, what could you do to make the project better? (LO\_ProbSolutions)
		\item Give a brief overview of other design solutions you considered.  What
		are the benefits and tradeoffs of those other designs compared with the chosen
		design?  From all the potential options, why did you select documented design?
		(LO\_Explores)
	\end{enumerate}
	
\end{document}
