\documentclass[12pt, titlepage]{article}

\usepackage{booktabs}
\usepackage{tabularx}
\usepackage{hyperref}
\hypersetup{
    colorlinks,
    citecolor=black,
    filecolor=black,
    linkcolor=red,
    urlcolor=blue
}
\usepackage[round]{natbib}

\input{../Comments}

\newcommand{\progname}{Software Eng 4G06} % PUT YOUR PROGRAM NAME HERE
\newcommand{\authname}{Team 2, Parnas' Pals
	\\ William Lee
	\\ Jared Bentvelsen
	\\ Bassel Rezkalla
	\\ Yuvraj Randhawa
	\\ Dimitri Tsampiras
	\\ Matthew McCracken} % AUTHOR NAMES                  

\usepackage{hyperref}
\hypersetup{colorlinks=true, linkcolor=blue, citecolor=blue, filecolor=blue,
	urlcolor=blue, unicode=false}
\urlstyle{same}



\begin{document}

\title{Verification and Validation Report: \progname} 
\author{\authname}
\date{\today}
	
\maketitle

\pagenumbering{roman}

\section{Revision History}

\begin{tabularx}{\textwidth}{p{3cm}p{2cm}X}
\toprule {\bf Date} & {\bf Version} & {\bf Notes}\\
\midrule
Date 1 & 1.0 & Notes\\
Date 2 & 1.1 & Notes\\
\bottomrule
\end{tabularx}

~\newpage

\section{Symbols, Abbreviations and Acronyms}

\renewcommand{\arraystretch}{1.2}
\begin{tabular}{l l} 
  \toprule		
  \textbf{symbol} & \textbf{description}\\
  \midrule 
  T & Test\\
  \bottomrule
\end{tabular}\\

\wss{symbols, abbreviations or acronyms -- you can reference the SRS tables if needed}

\newpage

\tableofcontents

\listoftables %if appropriate

\listoffigures %if appropriate

\newpage

\pagenumbering{arabic}

This document ...

\section{Functional Requirements Evaluation}

\section{Nonfunctional Requirements Evaluation}
Unless Specified these tests were performed with a semi-structured interview where users were brought scenarios then asked questions relevant to that scenario. Most Scenarios followed the key use cases of the application.\\
Examples include:
\begin{enumerate}
	\item Creating an Account.
	\item Logging into an account.
	\item Creating a Program.
	\item Creating a Workout.
	\item Creating an Exercise.
	\item Searching for an Exercise.
	\item Using the Discovery page to browse workouts and programs.
	\item Finding new workouts from the Discovery page.
	\item Starting a workout (as if the user were to use the application during their workout).
\end{enumerate}
\subsection{Look and Feel Testing}
\begin{enumerate}
	\item{\textbf{test-LF-1}}: Style.
\subsubsection{Usability and Humanity Tests}
	\item{\textbf{test-UH-1}}: Text Sizing and Visibility.
	\item{\textbf{test-UH-2}}: Text Language.
	\item{\textbf{test-UH-3}}: Learnability.
	\item{\textbf{test-UH-4}}: Understandability.
	\item{\textbf{test-UH-5}}: Hearing and Audio considerations.
	\item{\textbf{test-UH-6}}: Use of Colour and Contrast.
\subsubsection{Performance Tests}
	\item{\textbf{test-PF-1}}: Speed and Latency.
	\item{\textbf{test-PF-2}}: Accuracy and Precision of Quantifiers.
	\item{\textbf{test-PF-3}}: Availability and Uptime.
	\item{\textbf{test-PF-4}}: User Capacity.
	\item{\textbf{test-PF-5}}: Scalability of User Capacity.
\subsubsection{Operational and Environment Tests}
	\item{\textbf{test-OE-1}}: Supported Systems.
\subsubsection{Maintainability and Support Tests}
	\item{\textbf{test-MS-1}}: Maintenance.
\subsubsection{Security Tests}
	\item{\textbf{test-SEC-1}}: Private and Public Details.
	\item{\textbf{test-SEC-2}}: Passwords.
	\item{\textbf{test-SEC-3}}: Client Server Privacy.
	\item{\textbf{test-SEC-4}}: Data storage and logging.
	\item{\textbf{test-SEC-5}}: Data Backups.
	
\subsubsection{Cultural Requirements Tests}
	\item{\textbf{test-CR-1}}: Profanity and Inappropriate Language.
	\item{\textbf{test-CR-2}}: Reporting Offensive Language.
\subsubsection{Legal Requirements Tests}
	\item{\textbf{test-LR-1}}: Age and Gender Use.
	\item{\textbf{test-LR-2}}: Data Protection.
\end{enumerate}
	
\section{Comparison to Existing Implementation}	

This section will not be appropriate for every project.

\section{Unit Testing}

\section{Changes Due to Testing}

\section{Automated Testing}
		
\section{Trace to Requirements}
		
\section{Trace to Modules}		

\section{Code Coverage Metrics}

\bibliographystyle{plainnat}
\bibliography{../../refs/References}

\newpage{}
\section*{Appendix --- Reflection}

The information in this section will be used to evaluate the team members on the
graduate attribute of Lifelong Learning.  Please answer the following questions:

\begin{enumerate}
  \item 
  \item 
\end{enumerate}

\end{document}